\documentclass[12pt]{article}
\usepackage[a4paper]{geometry}

\usepackage{commath}

\usepackage[spanish]{babel}
\selectlanguage{spanish}
\usepackage[utf8]{inputenc}


%% Definitions of operators 
\newcommand{\ii}{\mathrm i}
\newcommand{\ee}{\mathrm e}
\DeclareMathOperator{\re}{Re}
\DeclareMathOperator{\im}{Im}
\DeclareMathOperator{\Arg}{Arg}
%\DeclareMathOperator{\arg}{arg}

\begin{document}


\noindent H.Rafeiro \hfill Variable Compleja \hfill Lección 2

\begin{enumerate}
	\item Hallar el \textit{argumento principal} $\Arg z$  y el \textit{argumento} $\arg z$ de: 
		\begin{enumerate}
			\item $z= \frac{\ii}{-2-2\ii}$
			\item $z= \left( \sqrt{3}-\ii \right)^6$
			\item $z=-1-\ii$
		\end{enumerate}

	\item Escribir en forma polar los siguientes números complejos:
		\begin{enumerate}
			%\item $ \cos 5\theta = 16 \cos^5 \theta - 20 \cos^3 \theta + 5 \cos \theta $ para aula 3 com formula de De Moivre
			\item $z= -1-\ii$
			\item $z= -\ii$
			\item $z=\left(\sqrt{ 3}+\ii\right)^7 $ 
			\item $z= \left( 2+2\ii \right)^{12}$
		\end{enumerate}
	\item Probar que
		\begin{enumerate}
		\item $|\ee^{\ii\theta}|=1$
		\item $\overline{\ee^{ \ii\theta}}= \ee^{-\ii \theta}$

		\end{enumerate}
\end{enumerate}

\end{document}
